% diagrams-Bd.tex
\begin{hcarentry}[updated]{diagrams}
\report{Brent Yorgey}%11/11
\status{active development}
\participants{Peter Hall, Andy Gill, Deepak Jois, Ian Ross, Michael Sloan, Ryan Yates}
\makeheader

The diagrams framework provides an embedded domain-specific language
for declarative drawing.  The overall vision is for diagrams to become
a viable alternative to DSLs like MetaPost or Asymptote, but with the
advantages of being \emph{declarative}---describing what to draw, not
how to draw it---and \emph{embedded}---putting the entire power of
Haskell (and Hackage) at the service of diagram creation.  There is
still much more to be done, but diagrams is already quite
fully-featured, with a comprehensive user manual, a large collection of
primitive shapes and attributes, many different modes of composition,
paths, cubic splines, images, text, arbitrary monoidal annotations,
named subdiagrams, and more.

%**<img width=500 src="./paradox.jpg">
%*ignore
\begin{center}
\includegraphics[width=0.47\textwidth]{html/paradox.jpg}
\end{center}
%*endignore

Since the previous HCAR, a new version of the framework has been
released, featuring experimental support for animations; a new package
of user-contributed modules, so far including tree drawing, Apollonian
gaskets, planar tilings, ``wrapped'' layout, and turtle graphics;
better performance; many other small additions and
improvements; and a redesigned website.

There is also a growing diagrams ``ecosystem''; related projects under
development include TikZ and HTML5 canvas backends, a Logo
interpreter, a graphing application, and a framework for creating
interactive GTK/cairo applications.

There is plenty more work to be done; new contributors are welcome!

%**<img width=500 src="./triangular-numbers.jpg">
%*ignore
\begin{center}
\includegraphics[width=0.47\textwidth]{html/triangular-numbers.jpg}
\end{center}
%*endignore

\FuturePlans

A native SVG backend is under active development and targeted for the
next release of the framework.  The cairo backend will still be
supported, but SVG will replace cairo as the default
``out-of-the-box'' backend, vastly simplifying installation for new
useres. Other plans for the near future include support for drawing
arrows and improvements to the handling of named subdiagrams.
Longer-term plans include support for interactive diagrams, a custom
Gtk application for editing diagrams, and any other awesome stuff we
think of.

\FurtherReading
\begin{compactitem}
\item \url{http://projects.haskell.org/diagrams}
\item \url{http://projects.haskell.org/diagrams/gallery.html}
\item \url{http://code.google.com/p/diagrams/issues/list}
\end{compactitem}
\end{hcarentry}
